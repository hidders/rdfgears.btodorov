
\chapter{\label{cha:conclusions}Conclusions and Future Work}

This chapter concludes our research. Section \ref{sec:concSumm} summarizes this thesis work and presents its main contributions. Section \ref{sec:concDisc} presents our conclusions on weather we have achieved the research goal. Finally Section \ref{sec:concFuture} discusses the possibilities for future work that we have identified.


\section{Summary}
\label{sec:concSumm}

In this thesis we presented the design and implementation of the U-Sem platform. Its main objective is to facilitate the everyday work of engineers building user modelling services based on the U-Sem idea. In order to achieve this objective, we first performed series of interviews in order to understand what are the tasks that engineers face every day and which of them require a lot of overhead work. Based on the results from the interviews we compiled a list of features which have to be implemented in order to achieve our goal. Due to the scope of this work we prioritized the features and designed and implemented only the ones that were indicated to bring the most improvement: the \textit{Plug-in Environment} and \textit{Data Management} features.

For the first feature we proposed and implemented a plug-in architecture based on OSGi which enables engineers to plug their custom RDF Gears function components and the related resources in the system on demand and without disturbing the work of other users of the system. We also proposed a mechanism for collaboration between engineers on component level using a plug-in repository. Finally, we designed and implemented a tool that facilitates the process of building plug-ins for the U-Sem platform and another tool that help engineers to manage the dependencies between plug-ins.

For the second feature we proposed and implemented a solution integrated in U-Sem that is capable of supporting the data management tasks in U-Sem. Based on Hibernate, it enables engineers to store, manipulate and perform complex queries on RGL data structures without having to program custom components and be aware of how and where data is actually stored. We also propose mechanism which enables engineers to collaborate on data level. Finally, we integrate the two features to completely automate the installation and configuration of plug-ins that provide U-Sem services containing data management components.

\section{Conclusion}
\label{sec:concDisc}

The research goal of this thesis project was defined as:\\

\textit{Design and implement a platform that facilitates the work of engineers building user modelling services following the U-Sem approach}\\

In order to evaluate whether this goal is achieved we performed experiments and asked engineers to test the proposed solution in real world scenarios. The results presented in Sections \ref{sec:evalPlugin} and \ref{sec:evalStorage} confirmed that the solution actually facilitates the process of building U-Sem services by reducing the number of overhead tasks that engineers have to perform enabling them to focus on the core of their work. It saves them time, efforts and also reduces the amount of knowledge required in order to work with the system which also facilitates the process of training new engineers. The results also revealed that the solution additionally improves the performance of engineers by enabling them to collaborate effectively with each other. Therefore, we are confident that we have achieved the research goal and the proposed solution actually serves its purpose.

Additionally, apart from the U-Sem context, the developments in this work might be also beneficial for other fields. On one hand, the system can also be considered as an extension of RDF Gears. Therefore, all engineers using RDF Gears can potentially benefit from this work. On the other hand, we have designed the proposed features to be loosely coupled from the workflow engine by designing them as separate sub-systems providing access through APIs. Therefore, this sub-systems can be easily integrated in other workflow engines and this may also facilitate the work of their users.

\section{Future work}
\label{sec:concFuture}

The possibilities for future work that we have identified can be split into two categories. The first one covers how each of the implemented features can be improved in the future. We already discussed these in the Sections \ref{sec:evalPlugin} and \ref{sec:evalStorage} dedicated for each of the features. 

The second category concerns the U-Sem platform as a whole. In Section \ref{sec:features} we have identified the features providing which will facilitate the work of the engineers. Due to the scope of this work, however, we designed and implemented only the ones that were indicated to be most beneficial. Even though the rest might not provide so much benefits we believe that designing and implementing them can still bring some important improvements for the everyday work of the engineers.
