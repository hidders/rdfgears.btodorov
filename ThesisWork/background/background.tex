
\chapter{\label{cha:background}Background}
This chapter discusses the background concepts that are needed in order to articulate the problem and also the related existing works. Section \ref{sec:userModellingSystems} provides a short introduction to the field of User Modelling Systems. In Section \ref{sec:usemFrm} we briefly introduce the idea behind the U-Sem framework. Finally, Section \ref{sec:workfloManagement} presents the idea of workflow engines and elaborates on RDF Gears in particular which is currently used to support the construction of the user modelling services.

\section{User Modelling Systems}
\label{sec:userModellingSystems}

User modelling is usually considered to originate from the work of Allen, Cohen and Perrault (\cite{allen1979plan}, \cite{cohen1979elements}) and Elaine Rich (\cite{rich1979building}, \cite{rich1979user}) \cite{kobsa2001generic}. Their work had inspired many engineers and scientists to build systems that collect information about their users and adapt specifically for each them \cite{wahlster1989user}. Initially, the user modelling functionality is integrated within the systems and there is little separation between the core and the user modelling functionality \cite{kobsa2001generic}. However, in the late 80s an effort is made to decouple the user modelling functionality from the rest of the user-adaptive systems and define it as separate components that can be reused \cite{kobsa2001generic}. 

In his work \cite{kobsa2001generic} classifies the user modelling components in two categories:
\begin{itemize}
	\item \textit{User Modelling Shells} - they are part of the system that makes use of the user modelling functionality. During the development phase of the system, they have to be integrated and configured specifically for the application they are used for.
	\item \textit{User Modelling Servers} - they are not part of the system but rather independent from it. Usually, they are "centralized", reside in the local area network or a wide area network and serve more than one applications at the same time. The main advantages of this approach lie in the fact that the information resides in a single place (no duplication and easier access control) and the functionality can be used by multiple applications which can also collaborate (information acquired by one application can be used by others).
\end{itemize}

A large number of user modelling servers have been developed over the time. Some of them are simple academic prototypes others are more sophisticated and commercially used. Examples of the foremost generic user modelling servers include DOPPELANGER \cite{orwant1994heterogeneous}, PersonisAD \cite{assad2007personisad}, Cumulate \cite{brusilovsky2004knowledgetree}, UMS \cite{kobsa2006ldap}.

However, with the increasing popularity of Social Web applications and the increasing amount of data that is published on the Web everyday there is a demand for a new type of user modelling servers that are specifically designed to take advantage of the available data on the Web \cite{brusilovsky2007adaptive}. In \cite{abel2011u} the authors address this issue and propose a special kind of framework for building user modelling servers based on various types of services. Next section discuses the idea behind it.

\section{The U-Sem framework}
\label{sec:usemFrm}

U-Sem represents a framework for semantic enrichment and mining of people's profiles from usage data on the Social Web. It provides methodology for designing services for user modelling and semantically augmenting user profiles. Its idea lies on the assumption that user modelling in today's Social Web sphere relies mainly on three types of input data \cite{abel2011u} that can be modelled in RDF format \cite{miller1998introduction}:

\begin{itemize}
	\item \textit{Observations} - usage data or events such as clicks, tagging activities, bookmarking actions or posts in social media.
	
	\item \textit{User characteristics} - explicit data provided by the users about their characteristics such as name, address, homepage, occupation, date of birth, etc.
	
	\item \textit{Domain knowledge} - this data is needed in order to produce user profiles that support certain application domains.
\end{itemize}

Based on these types of input data the authors propose that the user modelling services can be build out of various functional components. This components are classified in the following categories:

\begin{itemize}
	\item \textit{Semantic enrichment, Linkage and Alignment} - these components are responsible to process the input data. They provide functionality for aggregating and linking user data from Social Web systems like Facebook\footnote{Facebook - http://facebook.com/} and Twitter\footnote{Twitter - http://twitter.com/}, and integration and alignment of RDF data from Linked Data providers such as DBpedia\footnote{DBpedia - http://dbpedia.org/}.
	
	\item \textit{Analysis and User modelling} - components that given the enriched data from the components in the previous group perform the analysis and user modelling generating user profiles that describe interests, knowledge and other characteristics of the users.
\end{itemize}

Services are built by orchestrating these components into workflows that provide the required functionality. The process of creating and executing workflows is usually automated using workflow management systems which are discussed in the next section.

\section{Workflow Management Systems}
\label{sec:workfloManagement}

Workflow Management Systems are systems that allow a set of high level operators that all do a single well-defined specific task to be composed in a larger whole - the workflow. Depending on the desired application some systems are specifically concerned with  timing, concurrency, synchronization, data management aspects of a complex process. They may also imply special domain specific languages for defining the workflows. Literature provides a lot of examples of workflow management systems that are currently available: \textit{Taverna} \cite{hull2006taverna}, Kepler \cite{ludascher2006scientific}, RDFGears \cite{feliksik2011}, etc.

RDF Gears seems the best option for engineers applying the U-Sem framework because, firstly, it is specifically designed to deal with RDF data which is how all the user modelling inputs are modelled according to the specification of the U-Sem framework. Secondly, it is already being used by some engineers and it has proved to be suitable for the situation. In next section, we will discuss how RDF Gears works and what are its features.

\subsection{RDF Gears}
\label{sec:backRDFGears}

RDF Gears is a workflow management system designed specifically for building workflows for integration and transformation of RDF data \cite{feliksik2011}. It aims to provide a data integration framework that allows expressing and executing complex algorithms. RDF Gears consists of three main components: RDF Gears Language, RDF Gears Engine and RDF Gears graphical user interface \cite{feliksik2011}:

\paragraph{RDF Gears Language (RGL)} is a workflow language that is used to define the workflows as a combination of functional components. Each functional component has zero or more inputs and one output and defines certain operation that is performed on its inputs. The result of the operation is provided as the output of the component. The inputs are defined as input ports and can be connected with a particular output port of another component. All ports are associated with a data type. The type system of the language is discussed in next subsection. The connections between the components represent the flow of data between them. Components and the connections between them form an acyclic directed graph which is the actual workflow. 

\paragraph{RDF Gears graphical user interface} represents a web application that facilitates the construction of workflows in order to make the construction of complex RGL expression easier. It provides drag and drop operations for editing worklfows and visualizing the connections between components as a visual directed graph \cite{maro2011}. The GUI builds the defined workflows as XML files which are the workflow source code that represent the RGL expressions and are executed by the RDF Gears Engine.

\paragraph{RDF Gears Engine} represents an application that is responsible for the execution of the predefined workflows by evaluating the RGL expressions. Discussed in next subsection, it also provides optimizations for efficient execution of workflows. The engine expects as an input the XML files that define the workflows and provides graphical and RESTful interfaces that enable users to execute the workflows. 

RDF Gears also comes with a set of built-in components that can be used for creating workflows. However, this set is not exhaustive and usually engineers have to implement their custom components in order to be able to build the required user modelling services. The engine is implemented in Java and the process of adding custom components to it requires the following steps: First, engineers have to check out the source code of RDF Gears from the software repository. Then, they have to implement their components in Java and add the result Java classes and resources to the source code of the engine.  Finally, they have to build the system and deploy it on a Web Server so that it is available to the users.

\subsubsection{Type System}

The RGL language defines values and type systems. The values of RGL combine the value system of the Named Nested Relational Calculus (NNRC) with the RDF data model \cite{feliksik2011}. The RGL values can be defined as follows:

\begin{itemize}
	\item Every RDF value (URI, literal or graph) is a value in RGL.
	
	\item If r is a named row over values, then [r] is a value which is called a record.
	
	\item A finite multi set of values is a value that is called a bag. Bags are restricted to contain only values of the same type.
\end{itemize}

RGL uses static type system that allows engineers to specify the expected input and output values for each component on design time. It also allows the evaluation of the RGL expression to be performed before the actual execution in order to ensure the correctness of the RGL expression. RGL introduces types for the URI, literals and RDF graph values. We refer to them as basic types. RGL types are inductively defined as follows:

\begin{itemize}
	\item The basic types are types
	\item If T is a named row over types, the Record(T) is a type that defines record values
	\item If T is a type, then Bag(T) is a type that defines bag values
\end{itemize}

\subsubsection{Optimizations}

In order to execute the predefined workflows efficiently RDFGears performs special optimizations like pipelined bag iteration, lazy evaluation of expressions and buffered, streaming serialization \cite{feliksik2011}. Designing the solution proposed in this work we have to ensure that the current way for compiling and executing workflows is not compromised and also workflows are still optimized in order to execute efficiently. 

The way RDF Gears engine applies some of the optimizations is based on the assumption that wokrlfows are build from components that have no side effects. This means that they are only responsible to produce an output value and do not change their state or the state of any other component or system. As a result, if the output value of a component is not needed the engine can decide not to execute it at all and there will be no change in the behaviour of the workflow. The engine takes advantage of this idea in two aspects:
	\begin{itemize}
		\item \textit{Wokflow compilation} - workflows are compiled into a data structure which represents directed graph and contains the output node and all components that the output directly or indirectly depends on. Therefore, if there is no path in the graph from a component to the output node then the component is removed. The removed components are never executed.
		
		\item \textit{Workflow evaluation} - the engine provides functionality that performs lazy evaluation of the compiled workflows. The engine takes the output node and recursively executes all components which outputs are needed for the execution. As a result, components which output is not requested are not executed at all (eg. when the If/Then/Else component is used). Additionally, when a component's output is required by several components then the first time it is executed it cashes its result value and directly provides it for the next request. Therefore, RDF Gears guarantees that each component is executed at most once.
\end{itemize}
