With the increasing popularity of social media, more and more user data is published on the web everyday. As a result, there is a high demand for engineers to devise different algorithms for user modelling based on that data. The U-Sem framework defines an approach for constructing such algorithms and providing them to the customers in the form of user modelling services.

The process of building these services, however, requires engineers to perform a lot of manual tasks, many of which are not connected to the core of the engineers' specialization. These tasks are considered as significant overhead and are reported to cause poor performance of engineers. Therefore, in order to solve that problem, in this thesis we present the design and implementation of the U-Sem platform. It extends the U-Sem idea for building user modelling services by providing a platform that facilitates the work of the engineers.

Based on the analysis of the current process for building U-Sem services we solve the two problems which solutions were indicated to be the most beneficial for the engineers. The first one is to enable engineers to add and remove the functional components that build up the services to/from the platform on demand without affecting the work of others using it. While the second problem is to enable engineers to create and process the persistent data required for the services transparently without being aware of where and how it is actually stored.