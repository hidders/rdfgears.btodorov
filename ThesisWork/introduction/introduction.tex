
\chapter{\label{cha:intro}Introduction}

Nowadays, there are hundreds of millions of Internet users all over the world that use the internet in their work and leisure time. Every day they are sharing more and more content on the Web especially through the social media sites \cite{kaplan2010users}. This sharing is facilitated by many different systems: microblogging systems like Twitter\footnote{Twitter - https://twitter.com/}, picture sharing like Flickr\footnote{Flickr - http://www.flickr.com/}, professional profile sharing like LinkedIn\footnote{LinkedIn - http://www.linkedin.com/}, etc. As a result, there are many user traces on the Web \cite{abel2010interweaving}.

A lot of businesses and organizations are staring to understand how valuable this information can be and how they can benefit by extracting certain kinds of knowledge out of it. There are many fields that can benefit from this including recommender systems, advertising, e-learning, etc. As a result of the many possible applications there is a high demand for approaches and mechanisms for analysing this information and extracting the needed knowledge out of it. 

The field of user modelling and analysis is one of the fields that covers this area. It is considered to have a lot of potential \cite{brusilovsky2007adaptive} and therefore, there is a high demand for engineers to implement systems that provide new or adapt existing algorithms in order to meet the requirements of the customers. There are many existing approaches for building such systems \cite{kobsa2001generic} and one of the newly emerging ones is U-Sem \cite{abel2011u}. It is based on the idea that the user modelling functionality can be provided to the customers in the form of services which are built by orchestrating different functional components. This approach, however, focuses mainly on the organization and semantics of the building blocks of the services and pay little attention on how the systems are actually constructed. Currently, each engineer has adapted a personal approach for executing the tasks required for the construction of such systems. Many of these tasks, however, are not part of the core of the engineers' work. Such overhead tasks include setting up web servers and databases, implementing functionality that supports the interaction with the databases, communicating with fellow scientist in order to arrange reusing some of their work, etc. 

We believe that the current approach that engineers use for building user modelling services based on the U-Sem idea costs them a lot of time and efforts and requires them to spread their attention on a lot of areas many of which are not part of their speciality which may also influence negatively the quality of the final result. Therefore, in this thesis work we extend the U-Sem idea by proposing a user modelling platform that aims to facilitate the work of the engineers and improve their productivity by reducing the amount of the overhead work they have to perform so that they can focus on devising the modelling services which is also their main expertise.


\section{Research goal}

The research presented in this thesis was coordinated at TU Delft as part of ongoing research in the Web Information System (WIS) group. The main research goal of this work was formulated as follows:\\

\textit{\textbf{Research goal:} Design and implement a platform that facilitates the work of engineers building user modelling services following the U-Sem approach}\\

In order to understand the problem and achieve the research goal, there are some questions that need to be answered: What is the current process for building user modelling services following the U-Sem approach? Which are the tasks in this process that currently require engineers to invest a lot of time and efforts in? Which of these tasks are most beneficial to be facilitated? 

Based on the answers of these questions covered in Chapter \ref{cha:problemDef} we rephrased the research goal as extending the currently used system (RDF Gears) which facilitates the service orchestration task so that it also facilitates the other critical tasks involved in the process of building U-Sem services. Based on that, we formulated a set of sub-goals which lead to achieving the research goal posed in this thesis:

\begin{enumerate}
\item \textbf{Enable engineers to add and remove functional components to/from the system on demand without affecting the work of others using it.}
Functional components are the building blocks of the user modelling services. They are defined in the form of programming entities (e.g. Java classes) and the accompanying runtime resources.

\item \textbf{Enable engineers to create and process persistent data transparently without being aware of where and how it is actually stored.}
Processing persistent data implies that engineers are able to update, delete and query stored data.

\end{enumerate}

These goals have also defined the steps that we took to achieve our main research goal. Next section presents the outline of this work.


\section{Outline}
The structure of this document is as follows. First, Chapter 2 introduces the background information that is needed in order to be able to understand and articulate the problem.

In Chapter \ref{cha:problemDef} we define the problem that has to be solved and further decompose it in two separate sub problems. We also discuss the relevance of each of the sub-problems.

In Chapters 4 and 5 we address each of the sub-problems. This requires the identification of the formal requirements that has to be satisfied in order to solve the problems. We then present the approach taken for solving each of the problems which is based on the state of the art developments in the field. Then, we discuss the actual architecture of the solution and the way it is implemented. Finally, we evaluate the benefits that each of the solutions brings and propose directions for future work. 

Chapter 6 concludes this thesis. It provides a summary with conclusions and discusses future work.