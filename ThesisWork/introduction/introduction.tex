
\chapter{\label{cha:intro}Introduction}

Nowadays, there are hundreds of millions of Internet users all over the world that use the social media sites \cite{On social Web sites}. Every day they are sharing more and more content on the Web. This sharing is facilitated by many different systems: microblogging systems like Twitter, picture sharing like Flickr, professional profile sharing like LinkedIn, etc. As a result, there are a lot of user traces on the Web [2] which can be used for different applications such as adaptation and personalization[6].

Secondly, user modelling has become a really hot topic lately. Scientists are continuously coming up with new algorithms and approaches for analizing data and building user profiles. Most of the time, scientists are interested to easily make their work popular and available to a large auditory.


And finally, as a result of the increasing amounts of data available on the social web and the continuous advancements in the scientific approaches for user modeling the producese models has become interesting for many componies in variate of fields(recommender systems, advertisements, e-learning). However, because of the increased complexity of the modeling approaches and the continues inovations and improvements in the fields it has become infeasible for many organization(especially small companies) to implement and maintain user modelling functionality themselves. Therefore, many organizations are looking for ways to outsource the user modelling functionality to third parties.


In this thesis we are investigating how to bridge the gap between semantic web which provides the raw information needed for user modelling, scientists who want to popularize and provide access to their user modelling algorithms and clients which need the user models. 


\subsection{Research questions}

The main research question is the following:

\textbf{How to facilitate the work of scientists and organizations interested in user modelling and analysis?}\\

The following sub-questions articulate the problem:

\begin{enumerate}
\item \textbf{How to enable users to add custom functionality and change system's behaviour?} (plug-ins)

\item \textbf{Is it possible to develop a component that enables users to store and retrieve information in arbitrary data formats?} (Universal datastore)

\item \textbf{How to enable users to maintain their results up-to-date?}  (Scheduling)


\item \textbf{How to enable multiple users to use the system simultaneously without interfering with each other and keeping their work and results protected?} (Multi User and privacy)


\item \textbf{How can real-world use cases benefit from this system?}

\end{enumerate}

\subsection{Contributions}
This research provides the following contributions:

The first group of contributions concern the Plug-in environment feature.
\begin{itemize}
	\item Design and implementation of the dynamic component model for RDFGears based on OSGi.
	\item Design and implementation of the plug-in collaboration system.
	\item Design and implementation of a plug-in template that simplifies the creation of U-Sem plug-ins under Eclipse.
\end{itemize}

The second group concern the Data Management feature.
\begin{itemize}
	\item Design and implementation of the multi-tenant ORM framework for dynamic, virtual entities based on Hibernate.
	\item Extension for the RGL language and RDFGears engine that enables the system to deal with components that have side effects.
\end{itemize}

\subsection{Outline}


