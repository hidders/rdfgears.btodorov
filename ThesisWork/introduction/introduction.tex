
\chapter{\label{cha:intro}Introduction}

Nowadays, there are hundreds of millions of Internet users all over the world that use the internet in their work and leisure time. Every day they are sharing more and more content on the Web especially through the social media sites \cite{kaplan2010users}. This sharing is facilitated by many different systems: microblogging systems like Twitter\footnote{https://twitter.com/}, picture sharing like Flickr \footnote{http://www.flickr.com/}, professional profile sharing like LinkedIn\footnote{http://www.linkedin.com/}, etc. As a result, there are many user traces on the Web \cite{abel2010interweaving}.

A lot of businesses and fields are staring to understand how valuable this information can be and how they can benefit by extracting certain kinds of knowledge out of it. There are many areas that can benefit from this including recommender systems, advertising, e-learning, etc. As a result of the many possible applications there is a high demand for approaches and mechanisms for analysing this information and extracting the needed knowledge out of it. 

The field of user modelling and analysis is one of the fields that covers this area. It is considered to have a lot of potential \cite{brusilovsky2007adaptive} and therefore, there is a high demand for engineers to implement systems that provide new or adapt existing algorithms in order meet the requirements of the customers. There are many existing approaches for building such systems \cite{kobsa2001generic} and one of the newly emerging ones is U-Sem \cite{abel2011u}. It is based on the idea that the user modelling functionality can be provided to the customers in the form of services which are built by orchestrating different functional components. This approach, however, focuses mainly on the organization and semantics of the building blocks of the services and pay little attention on how the systems are actually constructed. Currently, each engineer has adapted a personal approach for executing the tasks required for the construction of the systems. Many of these tasks, however, are not part of the core of the engineers' work. Such overhead tasks include setting up web servers and databases, implementing functionality that supports the interaction with the databases, communicating with fellow scientist in order to arrange reusing some of their work, etc. 

We believe that the current approach that engineers use for building user modelling services based on the U-Sem idea costs them a lot of time and efforts and requires them to spread their attention on a lot of areas many of which are not part of their speciality which may also influence negatively the quality of the final result. Therefore, in this thesis work we extend the U-Sem idea by proposing a user modelling platform that aims to facilitate the work of the engineers and improve their productivity by reducing the amount of the overhead work they have to perform and focus on devising the modelling services which is also their main expertise.


\section{Research questions}

The main research question is the following:

\textbf{How to facilitate the work of engineers building user modelling services following the U-Sem approach?}\\

The following sub-questions articulate the problem:

\begin{enumerate}
\item \textbf{How to enable engineers to extend and modify the services provided by the system without affecting the other users using the system?}

\item \textbf{How to enable engineers to collaborate by reusing each others functionality?}

\item \textbf{How to enable engineers to work with persistent data without having to deal with issues connected to where and how it is stored?}

\item \textbf{How to enable engineers to collaborate on data level?}

\item \textbf{How can real-world use cases benefit from this system?}

\end{enumerate}


\section{Outline}
The structure of this document is as follows:

First, Chapter 2 introduces the background information that is needed in order to be able to understand and articulate the problem.

In Chapter 3 we define the problem that has to be solved and further decompose it in two separate sub problems. We also discuss the relevance of each of the sub-problems.

In Chapters 4 and 5 we address each of the sub-problems. This requires the identification of the formal requirements that has to be satisfied in order to solve the problems. We devise an approach for solving each of the problems that is based on the state of the art developments in the field. Then, we discuss the actual architecture of the solution and the way it is implemented. Finally, we evaluate the benefits that each of the solutions brings and propose some directions for future work. 

Chapter 6 concludes this thesis. It provides a summary with conclusions and discusses future work.